\documentclass[nochapterpage,bigchapter,linedtoc,longdoc,colorback,accentcolor=tud3b]{tudreport}

\usepackage[stable]{footmisc}
\usepackage[ngerman]{hyperref}

\usepackage{longtable}
\usepackage{multirow}
\usepackage{booktabs}

\hypersetup{%
  pdftitle={Making Matches - Recommending the right personality
  },
  pdfauthor={Paul Schweiger},
  pdfsubject={Personality Recommendation},
  pdfview=FitH,
  pdfstartview=FitV
}

\newlength{\longtablewidth}
\setlength{\longtablewidth}{0.7\linewidth}
\addtolength{\longtablewidth}{-\marginparsep}
\addtolength{\longtablewidth}{-\marginparwidth}


\title{Making Matches - Recommending the right personality}
\subtitle{Paul Schweiger}

\begin{document}
\maketitle

\chapter{Topic Summary}
As globalization and connectedness advance, online communities grow in importance and the offline world embraces online opportunities. A surplus of possible contacts emerges, making it challenging to have an optimal online experience by engaging with fitting personalities. This interaction between users is a cornerstone of many different activities, such as multiplayer gaming, social networks, crowdsourcing communities or online dating platforms, making matchmaking efforts important to further financial and social goals. Recommender Systems that account for an adequate operationalization of the relevant personality traits can be used to improve said goals all across different domains of usage.\\
The ever-growing market of multiplayer gaming experiences have created online spaces where different personalities are forced to work together. While forming an effective team is the apparent goal of competitive online gaming, for the most part of the community \textit{having fun} is even more important. At the same time, the internet's anonymity makes it a great place for trolls and toxic players, resulting in intra-community problems and a “salty” experience. Not only toxic players, but a combination of different player goals and styles can lead to bad experiences: a single aggressive player in a defensive team will not have the needed support in battle, while players aiming for a story-driven experience won’t be able to cooperate with performance-oriented hardcore gamers.\\
Considering the fact that a large amount of players is available at any given moment, it should be possible to incorporate such preferences in the matchmaking process - which is currently mainly driven by player experience and performance. \\
A recent study by Wang et al. \cite{wang2015thinking} looked at the enjoyment of multiplayer sessions in League of Legends (LoL) based on player personality. They followed a subset of Sternberg's \cite{sternberg1999thinking} problem-solving styles in order to categorize the playstyle of different users. While others tried to gain player personality knowledge by simply asking them \cite{riegelsberger2007personality} or other players \cite{patrick2011system}, Wang et al. automated this process with gameplay statistics. They assigned specific in-game actions to each of their player categories and determined a player's style based on his action profile. Choosing the data used to calculate a "good" experience is highly domain specific and depends on available objective game statistics.\cite{delalleau2012beyond} Match enjoyment for LoL was modeled as a function of game length, with shorter games being less enjoyable due to one team dominating the other.  The results show a clear tendency of specific globally-active and risk-taking players to positively influence the overall game enjoyment. A neural network tasked with predicting match enjoyment achieved better scores when player style data was included, suggesting that this might benefit matchmaking algorithms. However, it is important to note that the used measures are highly subjective to LoL, as interviews with players of both LoL and other games confirm.\\
Most Online Dating Services experience comparable problems: Learning who goes good with one another is a major success criterion for dating applications. Possible matchups need to be built according to their preferences rather based upon global performance. Studies predicting user tastes based on the tastes of similar users, a technique called "Collaborative Filtering", showed better results than global matchmaking. \cite{brozovsky2007recommender}\\
Proper personality-based matchmaking in online environments has a positive influence on overall user satisfaction. Successful Recommender Systems need to incorporate personal and detailed data, rather than just performance measurements. On the other hand the limiting effect of Recommender Systems on content diversity needs to be considered, especially in  this social use case. \cite{nguyen2014exploring}\\

\chapter{Introduction}
The modern world has a plethora of opportunities, topics, people, products, and other things, while people have a limited amount of attention and time to spend. In order to help us guide our attention and resources, basically every digital product tries to recommend meaningful content to users. Recommender Systems play a critical role in modern society and have transcended in many different domains.\\
Online communities grow in importance and the offline world embraces online opportunities. A surplus of possible social contacts emerges, making it challenging to engage with the right person at the right time. 
Interaction between users is a cornerstone of many different activities, such as multiplayer gaming, social networks, crowdsourcing communities or online dating platforms. But besides entertainment, social contacts can also further educational goals.\\
SOZIALES LERNEN IST BESSER als Alleine [Quellen]\\
E-Learning and Online Courses -- Wo passt das sinnvoll hin? :S\\
Social Recommendation is more complicated, since it needs to account for preferences of more than just a single person. Recommender Systems need account for an adequate operationalization of the relevant personality traits, domain-specific skills and surrounding circumstances.\\
\\
Focus on E-Learning and Reciprocal Recommendation, because...\\
Introduction to Recommendation in E-Learning (How good is research? Overlap with other topics -> shorten this section)\\
Introduction to Recommendation in Reciprocal environments (How good is this researched? How many things do we know?)\\
This report will discuss possible solutions to improve learning via reciprocal peer recommendation in Learning-Environments. As one of the first (???) and the most recent implementation of such an endeavor, the report will provide a detailed overview of XXXXXXXXX by XXXXXXXX. Important findings leading up towards the work, possible extensions and research in other fields that might prove to be beneficial to this study will be discussed.\\

\chapter{Research leading towards the paper}
\section{Learning}
TODO: Provide overview of modern learning:\\
\begin{itemize}
	\item learner types
	\item E-Learning vs. offline learning
	\item peer learning
	\item recommender systems for e-learning so far
\end{itemize}

\section{Recommender Systems}
TODO: Provide short overview of RS\\
Where do they come from, what have they been used for?\\
REMARK: This section might be removed due to overlap with other sections. Maybe RS for E-Learning and reciprocal RS will be enough\\

\section{Reciprocal Recommendation}
TODO: Provide Overview of Reciprocal Recommendation\\
\begin{itemize}
	\item Goal of recirpocal recommendation
	\item different ways to do this: Just a score for other people, or actual reciprocal recommendations, where bothpartners see each other as a recom.
	\item Other fields (Gaming, Dating)
	\item Common methods (Collaborative Filtering)
	\item Common problems encountered when doing this (operationalize personality / relevant aspects, acquire personal information about people, self-reported vs. implicit, ...)
\end{itemize}

\chapter{Reciprocal Peer Recommendation for Learning Purposes}
\section{Introduction}
Introduction: Motivation for this specific study. What did they want to accomplish?\\
Explain goals and problems of 
XXX et al. implemented Ripple [capitalization!] to prove their theories. 
\section{Ripple}
Explain Ripples aspects and assumptions and how people get matched.\\
Ripple will be evaluated in live conditions in the course of this year. To check whether the implementation could work under real conditions, XXXX conducted an evaluation using randomly generated data. This will be discussed in the upcoming section\\

\section{Evaluation}
In order to test RiPPLE's applicability for actual use, Potts et al. designed an experimental setup in which RiPPLE would try to propose recommendations for randomly generated test data. Specific quality measures were designed to assess different fields in which RiPPLE would have to show its capabilities. With satisfying results, RiPPLE would be able to be used under live conditionsin the course of 2018.\\

\subsection{Data}
To conduct the experimental evaluation, random data had to be generated; diverse enough to highlight edge cases but within reasonable bounds. For a set of 1000 users, 10 learning topics and 10 possible timeslots, each user received a random distribution of relevant values. Topic-specific competencies were expressed as a value from 0 to 100 derived from a truncated normal distribution around a random mean with fixed variance. Each user received competencies for every topic. These were then sorted from low to high, and a random number of these topics were chosen to be part of the user's request. The highest competencies of every user were classified as "providing support" roles, the lowest as "seeking support" and the median topic received a "co-study" role. Every "user" was made available during random timeslots.\\
LETZTER TEIL NOCH NICHT GANZ KLAR! WIE WIRD BESTIMMT WIE VIEL UNTERSCHIED IN KOMPETENZ GUT IST??\\

\subsection{Quality Measures}
As Evaluation Metrics for their experimental evaluation Potts et al. decided on four values that can further be used as general Quality Measures for reciprocal recommendation algorithms:\\
The overall results were considered to be promising for usage with real data and provided some input on how variables are connected and in which cases issues could arise from real-world data.\\
Mit Ergebnissen zusammenfassen oder lieber hier nur Namedroppen und unten genauer erklären?\\	
\subsubsection{Scalability} The potential suitability of the algorithm for large amounts of data. High runtime and costs for evaluating datasets with reasonable amounts of students means slower responses and a worse user experience. An optimal solution could provide immediate recommendations to any user, at any moment. Alternatively, Ripple could be executed once per week and provide suggestions on the user's next login.\\
The runtime of the algorithm increased in a quadratic fashion, as U, the total amount of users, increased: \(O(n^2)\). The number of recommendations per user, however, does not significantly impact the runtime. [DAS IST MEINE EINSCHÄTZUNG AUFGRUND DER DATEN!].\\
Currently, RiPPLE calculates recommendations at the end of each week for the upcoming week, making the algorithm's runtime rather unimportant. However, further improvements are planned.\\
In a 1000 user experiment, RiPPLE was able to provide recommendations for a single user in 0.045 seconds - which is a pretty good time.\\
TODO: BILDER
\subsubsection{Reciprocality} The best possible recommendations are reciprocal: Users contacting a recommended user would also appear on this user's list of potential study partners. To test reciprocality, a baseline non-reciprocal score [HIER NOCH MAL REINSEHEN! DAS IST MIR UNKLAR!] was compared to the precision of reciprocal recommendations: the precision for every user is calculated by dividing his reciprocal recommendations through the total amount of recommendations k that user received. The system's total precision is defined as the average precision across all users.\\ HIER UNBEDINGT NOCH QUELLE ANBRINGEN!
In all tested cases, the reciprocal score had a higher precision than the baseline score. Increasing \(k\) also increases the precision, since more recommendations per user lead to a higher chance of reciprocal recommendations. On the other hand, increasing \(U\) with a fixed \(k\) reduces reciprocal precision, since there are more possible users to recommend.\\
TODO: BILDER
\subsubsection{Coverage} Recommending potential learning partners to one another should be beneficial to as many students as possible. As such, coverage is a very important metric to consider. Since every user will receive recommendations no matter what, every user will be covered in one way or another. A good fit can only be ensured when the same user is recommended to others, ideally forming a reciprocal recommendation. Coverage is defined as the percentage of users that appear on other's recommendations at least once.\\
For a low amount of users and lots of recommendations per user, coverage is close to 0.9, meaning most users are recommended to others. As U increases or k decreases, the coverage sinks. However, more than 40\% of users appear in other's recommendations under all tested circumstances.\\
\subsubsection{Quality} [NOCH MAL GENAUER VERSTEHEN! T UND ALPHA!] The quality of a recommendation is not only based on its fit, but also on how good the resulting team could perform. The quality is thus defined as their average joint competencies across their matched topics. The goal is to generate matches, that are somewhat capable in their respective fields of study.\\
To fully satisfy as a tool to recommend students to one another, Ripple must score as high as possible in these measures. On the other hand, minor drawbacks in the defined metrics were considered to be tolerable due to the experimental and randomly generated data and some further adjustments that could be made to compensate bad values.\\
RESULTS SIND MIR HIER MEGA UNKLAR - NACHARBEITEN!

\section{Discussion}
Recap of their discussion\\
Current results are scientifically weak: Lots of assumptions about initial values (e.g. for T), no aims of bounds set to determine whether final values are actually good.\\
While the values reported from evaluation with artificial data present good metrics to measure the algorithm's performance and suitability for live data, they don't actually evaluate the algorithm, since no targets have been set. The question whether a coverage of little above 0.4 will be enough in practice, remains unanswered. Same goes for all the other values: With little to no theory behind these metrics, their final values are hard to analyze in a larger context.\\
\\
Is the quality a good measurement, actually? We want low level learners to have the opportunity to learn something... Should the score of a match not be always close to a medium value? OR does this metric rely too much on the actual values of their skills? Should me measure quality as "How does this matchup go in terms of preferred skill level?\\
\\
Another neglected factor is the human factor. Both user buy-in and competence in handling the tool and its demands might influence its use in practice. While this study's goal was explicitly to test the theory and future praxis tests are planned, this topic should be discussed, a major shortcoming of the paper at hand.\\
A lack in user buy-in is something that always should be considered, especially in a student context. If a student didn't want to engage with foreign people, was not motivated to study with partners or to adjust his or her schedule, all recommendations to and of that student would not accomplish anything. Meeting requests would be ignored, and opportunities for matchups would expire. Even user manipulation needs to be considered as an possibility, but is something that has to be dealt with online.\\
While missed opportunities are a problem of the students themselves, rather than of the platform providing recommendations, the other human factor needs to be addressed directly by the tool.\\
As humans are unreliable, self-reported metrics always underly lots of variance and errors. A user's competency in a specific topic, his or her preferences, or the willingness to commit a specific timeslot to learning might change daily, dependent on mood, time of day and lots of other factors. [QUELLE?] Other variables, like a user's preferred skill difference towards a learning partner, are especially hard to specify. How is a user supposed to know what his or her learning preferences are? How would he know which number refers to the desired difference in skill rating? From a psychological standpoint, this operationalization is bound to fail.\\
\\
Does not consider buy-in by users. This is okay for the tool, it will still recommend people, but the actual use of this whole measure will get undermined.\\
self-evaluating data about students is prone to errors and was not checked or detailed: misleading competency values might spoil results.\\
Abandoned users due to edge-cases: Someone with low competency prefering to teach, high competency preferring to get teached. Ripple does account for these cases by only recommending the best fits, so even low fits can still be recommended.\\
Transparency: Will students know how good their matching value is?\\
Possible error for highly compatible users who appear in many recommendations: They have few reciprocal recs, but will get lots of meeting requests.\\
ARE THERE ANY DISCUSSIONS ABOUT THIS PAPER YET?\\


\chapter{Extensions}
TODO: Include possible follow-up research, directions for further research or innovative ideas from other fields to include in future recirpocal peer recommendation for elearning studies.\\
\begin{itemize}
	\item NEO-FFI etc.: operationalize personalities
	\item Instead of student-driven requests: Use recommendations to build study courses
	\item From Gaming: Instead of 2-people-matches, build learning groups with matching skillsets to further benefit on other topics and to enable social grouping\\
\end{itemize}


\bibliographystyle{alpha}
\bibliography{bib}

\end{document}
